\documentclass[a4paper,10pt]{article}
% Paquete para inclusión de gráficos.
\usepackage{graphicx}
% Paquete para definir el idioma usado.
\usepackage[spanish]{babel}
% Paquete para definir la codificación del conjunto de caracteres usado
% (latin1 es ISO 8859-1).
\usepackage[latin1]{inputenc}
\usepackage{hyperref}
% Include the listings-package
\usepackage{listings}  
\usepackage{pdfpages}


% T�tulo principal del documento.
\title{	\ Trabajo Práctico Nro 0: Infraestructura Básica}
% Información sobre los autores.
\author{    Nestor Huallpa, \textit{Padrón Nro. 88614}\\
            \texttt{ huallpa.nestor@gmail.com }\\\\              
            \texttt{\footnotesize 1er Entrega: 25/08/2018}\\
            \\\\\\\\\\\\\\\\\\
            \normalsize{2do. Cuatrimestre de 2018}\\ 
            \normalsize{66.20 Organización de Computadoras} \\
            \normalsize{Facultad de Ingeniería, Universidad de Buenos Aires} \\}
       
\date{}

\begin{document}
% Inserta el título.
\maketitle
% quita el número en la primer página
\thispagestyle{empty}
% Resumen
\begin{abstract}
En el presente trabajo práctico se describirán todos los pasos y 
conclusiones relacionadas al desarrollo e implementación de la codificacion y decodificacion de datos formateados en base 64.
\end{abstract}
\newpage{}
\tableofcontents
\newpage{}

\begin{flushleft}

\par\end{flushleft}
\section{{\normalsize Introducción}}

El objetivo del presente trabajo práctico es familiarizarse con el emulador gxemul mediante la implementacion de un programa que codifica y decodifica datos en base 64.



\bibliographystyle{plain}
\nocite{*}
\end{document}
